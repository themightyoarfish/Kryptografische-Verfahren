\documentclass{../crypto}
\sheet{4}
\date{27. November 2015}

\begin{document}

\maketitle

\section{Zufallsgeneratoren}

Meines Erachtens macht es keinen Sinn, nach einem generellen Beweis zu fragen,
dass $L$ und $H$ keine PZG sind. Eher sollte es darum gehen, dass sie nicht
\emph{notwendigerweise} PZG sind, wenn $G$ und $G^\prime$ PZG sind, denn dies
könnte von der Wahl von $G$ und $G^\prime$ abhängen. Diese
Interpretation der Aufgabe wird hier angenommen (so wie z.B. Aufgabe 3.6 aus
Katz \& Lindell).

Sei $G$ ein PZG mit $|G(s)| > 4|s| = 4n$. Definiere $G^\prime \coloneqq
G(s_1,\ldots,s_{\floor{ \frac{n}{2} }})$.  Nach Voraussetzungen ist
$G^\prime$ PZG, denn $|G(s)| > 2n$. 

\subsection{}

Da $|G^\prime| > 2n$, gilt 
\begin{equation*}
   H(s) = G^\prime(0^{|s|} ~||~ s) = G(0^{|s|})
\end{equation*}
Also berechnet $H$ für alle $s$ der Länge $n$ den selben Wert und ist damit leicht von echtem
Zufall unterscheidbar. Ein Distinguisher $\mcal{D}$ muss nur für gegebenes $|s|$
(die Länge des Seeds dürfte bekannt sein, oder er probiert alle $n$ Längen
durch, was polynomiell in $n$ wäre)
den Wert $G(0^{|s|})$ berechnen und prüfen, ob $H(s) = G(0^{|s|})$. Daher ist
$P(\mcal{D}(G(s)) = 1) = 1$, während die Situation, dass ein zufälliger
Bistring $r$ gerade $r=G(0^{|s|})$ ist, nur eine von $2^{|s|}$ Möglichkeiten
ist, $r$ zu wählen, mithin $P(\mcal{D}(r) = 1) = \frac{1}{2^{|s|}}$. Die
Wahrscheinlichkeiten unterscheiden sich also nicht-vernachlässigbar.

\subsection{}

% Not sure if this suffices; Maybe one could use ideas from
% https://crypto.stackexchange.com/questions/30698/given-prng-gs-why-is-gs-gs1-not-a-prng,
% but the guy/girl makes an assumption about some construction being a PRNG

% B-B-B-B-Bullshiiiiiiet!
% Ein Distinguisher $\mcal{D}$ berechnet $L(s)$ und $L(s+1)$ und gibt $1$ aus,
% sofern die ersten $\frac{l(n)}{2}$ Bits von $L(s+1)$  übereinstimmen mit den
% letzten $\frac{l(n)}{2}$ Bits von $L(s)$. Ein von $L$ generierter String wird
% also mit Wahrscheinlichkeit $1$ als solcher erkannt. Hingegen ist die
% Wahrscheinlichkeit, dass bei echt zufälligem $G(s+1)$ die erste Hälfte gleich der
% zweiten von $G(s)$ ist $\frac{1}{2^{\frac{l(n)}{2}}}\cdot \frac{1}{2^{\frac{l(n)}{2}}} =
% \frac{1}{2^{\frac{l(n)}{2}}}$. Die Wahrscheinlichkeiten unterscheiden sich
% nicht-vernachlässigbar.

Da $|G^\prime| > 2n$, nehmen wir an 
\begin{align*}
   L(s) & = G^\prime(s) ~||~ G^\prime(s+1)  \\
        & = G(s_1,\ldots,s_{\floor{\frac{n}{2}}}) ~||~ G^\prime(s+1) \\
        & = G(s_1,\ldots,s_{\floor{\frac{n}{2}}}) ~||~ G(s^\prime) \\
        & = G(s_1,\ldots,s_{\floor{\frac{n}{2}}}) ~||~ G(s^\prime_1,\ldots,s^\prime_{\floor{\frac{n}{2}}}) \\
\end{align*}

Sei $D(K)$ ein Distinguisher mit 
\begin{equation*}
   D(K) = 
   \begin{cases}
      1, \text{falls $k_1,\ldots,k_{\frac{n}{2}} = k_{\frac{n}{2} + 1},\ldots,k_n$} \\
      0\ \text{sonst}
   \end{cases}
\end{equation*}

$|K|$ muss für Strings, die durch $L$ generiert wurden, gerade sein, da es
eine Konkatenation zweier gleich langer Strings ist. Strings ungerader Länge
können sofort als echt zufällig identifiziert werden und sind daher
uninteressant.

Addiert man $1$ zu $s$, so ist die Wahrscheinlichkeit, dass sich ein Bit in der
vorderen Hälfte ändert die Wahrscheinlichkeit, dass alle Bits der unteren Hälfte
$1$ sind. Damit $D$ $0$ zurück gibt auf einem von $L$ generierten String, muss
genau der Fall eintreten, da die untere Hälfte jeweils abgeschnitten und
ignoriert wurde.
\begin{equation*}
   P(\text{Bit $i$ geflippt mit $i>\frac{n}{2}$}) = \frac{1}{2^{\frac{n}{2}}},
\end{equation*}
denn die unteren $\frac{n}{2}$ Bits müssen $1$ sein.
Daher ist 
\begin{equation*}
   P(D(G(s)) = 1) = 1 - P(D(G(s)) = 0) = 1 - \frac{1}{2^\frac{n}{2}}
\end{equation*}

Die Wahrscheinlichkeit, dass bei einem gleichverteilten Bitstring $r$ die obere
und untere Hälfte übereinstimmen (was zu einer Fehleinschätzung von $D(r)$
führen würde), lässt sich als Quotient der Anzahl von Strings, deren Hälften
gleich sind und der Anzahl aller Strings berechnen. Für eine Hälfte gibt es
$2^{\frac{n}{2}}$ Möglichkeiten und daher genauso viele Strings mit gleichen
Hälften, gegenüber $2^n$ Strings insgesamt.
\begin{equation*}
   P(D(r) = 1) = \frac{2^{\frac{n}{2}}}{2^n} = 2^{-\frac{n}{2}} = \frac{1}{2^\frac{n}{2}}
\end{equation*}

Es gilt also 
\begin{equation*}
   \left\lvert P(D(G(s)) = 1) - P(D(r) = 1) \right\rvert = \left\lvert1 - \frac{1}{2^\frac{n}{2}} -
   \frac{1}{2^\frac{n}{2}}\right\rvert = 1 - \frac{2}{2^{\frac{n}{2}}} = 1 -
   \frac{1}{2^{\frac{n}{2} - 1}}
\end{equation*}

Dies ist nicht vernachlässigbar und wird insbesondere größer für wachsendes $n$.
$L$ ist also nicht notwendigerweise ein PZG.

\section{Beweis des Satzes von Goldbach-Levin (erster Teil)}

Unter der Annahme, dass $g$ keine Einwegfunktion ist---also einen nicht
vernachlässigbar häufig erfolgreichen Angreifer $\mcal{A}^\prime$
besitzt---können wir einen Angreifer $\mcal{A}$ auf die eigentliche
Einwegfunktion $f$ konstruieren. $\mcal{A}$ agiert als Orakel für
$\mcal{A}^\prime$ und sendet seine erhaltene Eingabe $y$ weiter an
$\mcal{A}^\prime$, nachdem es ein beliebiges $x_2$ mit $|x_2| = |y|$ gewählt
hat. $\mcal{A}^\prime$ antwortet mit einer Lösung $(x_1^\prime,x_2)$ für das
Tupel. $\mcal{A}$ kann nun $x_1^\prime$  weiterleiten an $f$. 

\begin{center}
   \begin{tikzpicture}[node distance=4cm]
      \node[box,fill=black,text=white] (A') {$\mcal{A}^\prime$ auf $g$};
      \node[box,right=of A'] (A) {$\mcal{A}$ auf $f$};
      \node[box,right=of A] (f) {$f(x)$};

      \draw[thick,>=latex,<-] (A'.30)     -- node[above] {$(x_1=y,x_2)$} (A.150);
      \draw[thick,>=latex,<-] (A.30)    -- node[above] {$y=f(x)$} (f.150);
      \draw[thick,>=latex,<-] (f.210) -- node[below] {$x_1^\prime$} (A.-30);
      \draw[thick,>=latex,<-] (A.210)   -- node[below] {$(x_1^\prime,x_2)$} (A'.-30);
   \end{tikzpicture}
\end{center}

Offensichtlich ist $\mcal{A}$ genau dann
erfolgreich, wenn $\mcal{A}^\prime$ erfolgreich ist. Dies ist ein Widerspruch
zur Annahme, dass $f$ eine Einwegfunktion ist.

\hfill{}$\Box$

% Put up your damn slides, man.

\section{Einwegfunktionen?}

Angenommen, $g$ wäre keine Einwegfunktion und man könnte $g^{-1}$ effizient
berechnen. Dann gälte 
\begin{align*}
   g^{-1}(g(x)) &= x \\
   g^{-1}(f(f(x))) &= x \\
   g^{-1}(y) &= f^{-1}(f^{-1}(y)) \\
   \intertext{
      $g^{-1}$ revidiert also zwei Anwendungen von $f$. Dann ist aber auch
      für $y=f(x)$
   }
   g^{-1}(f(y)) &= f^{-1}(f^{-1}(f(y))) \\
                &= f^{-1}(y)
\end{align*}
Und die Umkehrfunktion $f^{-1}(y)$ ist effizient als $g^{-1}(f(y))$ berechenbar,
was ein Widerspruch ist, denn $f$ ist Einwegfunktion. $g$ ist also eine
Einwegfunktion.

Ich vermute, dass $g^\prime$ ebenfalls eine Einwegfunkion ist. Gegeben ein $y$
könnte man die Konkatenationsstelle zwar leicht bestimmen durch sukzessive
Anwendung von $f$ auf immer längere Teilstrings und das Problem so in zwei Teile
brechen, jedoch müsste man dann $f(x)$ und $f(f(x)) = g(x)$ invertieren. Wir
wissen, dass wir $g^{-1}$ allein nicht effizient bestimmen können. Möglicherweise ist
es hilfreich, für ein $f(f(x))$ das Urbild zu kennen (da es am Anfang steht),
aber ich sehe keinen Weg, dies zur Berechnung von $x$ zu nutzen.
\end{document}
