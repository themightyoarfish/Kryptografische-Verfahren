\documentclass{../crypto}
\sheet{1}
\date{22. Oktober 2015}

\begin{document}

\maketitle


\section{Rechnen mit Resten}

\subsection{}

  \begin{IEEEeqnarray*}{rCl}
     [a\cdot b]_n  & \stackrel{?}{=} & [a]_n \odot_n [b]_n \\
                   & =               & [(n\cdot q + r) \cdot (n\cdot p + s)]_n \\
                   & =               & [n^2pq + nqs + npr + rs]_n \\
                   & =               & [n^2pq]_n \oplus_n [nqs]_n \oplus_n [npr]_n \oplus_n [rs]_n \\
                   & =               & [rs]_n \\
                   & =               & r \odot_n s \\
                   & =               & [a]_n \odot_n [b]_n
  \end{IEEEeqnarray*}

\subsection{}

  \begin{IEEEeqnarray*}{rCl}
     \left(\left(3126 \oplus_{11} 21783\right) \odot_{11} 213894\right)
     \oplus_{11} 31942 & = & \left(\left[24909\right]_{11} \odot_{11} 213894\right) \oplus_{11} 31942 \\
                       & = & \left(5 \odot_{11} 213894\right) \oplus_{11} 31942 \\
                       & = & \left[1069470\right]_{11} \oplus_{11} 31942 \\
                       & = & 6 \oplus_{11} 31942 \\
                       & = & \left[31948\right]_{11} \\
                       & = & 4
  \end{IEEEeqnarray*}

\subsection{}

  \begin{IEEEeqnarray*}{rCltt}
    \left[2131897^8\right]_3   & = & \left[\prod_{i=1}^8 2131897\right]_3\\
                    & = & \prod_{i=1}^8 \left[2131897\right]_3   & \hspace{1cm} & Aufgabe 1.1.~(a) \\
                    & = & \prod_{i=1}^8 1             & \hspace{1cm} &
    \begin{IEEEeqnarraybox}[][c]{s}
       Es sind $21\cdot x$, $3\cdot x$, $18\cdot x$ \\
       und $9\cdot x$ durch $3$ teilbar, \\
       sodass nur $7$ für Rest sorgen kann\\
    \end{IEEEeqnarraybox} \\
                   & = & 1
  \end{IEEEeqnarray*}


\section{Die Skytale}

\subsection{}

\lstinputlisting[language=Python]{skytale.py}
Die Lösung ist: \texttt{PIESKYTALEISTEINTRANSPOSITIONSVERFAHREN}, also in etwa
\texttt{Die Skytale ist ein Transpositionsverfahren}.

\subsection{}

Für die Berechnung der \texttt{pos} im Java Code muss noch der Parameter $k$
bekannt sein.

Angenommen der Index in der \texttt{for}-loop soll eigentlich in \texttt{i}
sein, ergibt sich die Formel:

\begin{align*}
\text{\texttt{pos}} = \left[i\right]_kw + \left\lfloor\frac{i}{k}\right\rfloor
\end{align*}

Die Intuition hier ist: Gehe Spalte für Spalte durch das Array, das $0$-te
Element zuerst, danach das $w$-te, dann das $2w$-te und so weiter. Daraus folgt
\texttt{pos} $= iw$. Damit nach Erreichen der letzten Zeile wieder die erste
verwendet wird, muss $i$ alle $k$ Schritte zurückgesetzt werden (nur für die
Berechnung, nicht als Loopindex), d.h. \texttt{pos} $= \left[i\right]_kw$. So
würde jedoch immer nur die erste Spalte ausgelesen. Um zusätzlich die Spalte zu
erhöhen, wenn das Ende erreicht ist, kann $\left\lfloor\frac{i}{k}\right\rfloor$
addiert werden, was den Spaltenindex für den Loopindex $i$ bei Spalten mit $k$
Elementen angibt. Es folgt $\texttt{pos} = \left[i\right]_kw +
\left\lfloor\frac{i}{k}\right\rfloor$.

Vollständig implementiert folgt:
\lstinputlisting[language=Java,firstline=9,lastline=12 ]{skytale.java}
\texttt{PIESKYTALEISTEINTRANSPOSITIONSVERFAHREN} wird korrekt ($k=5$) zum Kryptotext aus Aufgabe 1.2.~(a).

\subsection{}
Es kann passieren, dass die letzte Zeile nicht vollständig oder gar überhaupt
nicht gefüllt wird. Im Beispiel aus Aufgabe 1.2.~(a) ist dies kein Problem, ein
Zeichen ist in Ordnung.  Doch wenn in der letzten Zeile mehrere Zeichen fehlen
(oder gar mehrere Zeichen in der letzten Spalte, z.B. bei $|M|=41, k=8$) dann
gibt es ein Problem.

Genau dies ist der Fall für $|M|=42, k=8$. Es folgt $w=6$. Tatsächlich ist $|M|$
ein Vielfaches von $w$: $7w=|M|$. Das bedeutet, es werden nur 7 Zeilen gefüllt,
die unterste Zeile bleibt leer.

Das würde in der Praxis bedeuten, dass das Band mit Leerzeichen aufgefüllt wird
(unter Umständen mit Ausnahme des letzten Zeichens). Natürlich könnte auch ein
beliebiges anderes Zeichen gewählt werden, nennen wir dieses Zeichen $x$. Nun
kann ein Angreifer versuchen durch geschickes anordnen gleicher Buchstaben
nebeneinander ohne den Schlüssel zu kennen die Nachricht zu entschlüsseln.

Um dieses Problem zu umgehen gibt es zwei einfache Möglichkeiten:

\begin{enumerate}
  \item Klartext-Schlüssel-Kombinationen wählen, die möglichst gut $|M|=kw$ annähern.
  \item Den Klartext mit zufälligen Zeichen auffüllen, bis $|M|=kw$ annähernd gilt.
\end{enumerate}

Ein weiteres nicht mit der Skytale lösbares Problem ist, dass der Angreifer
durch Faktorisierungen der Zahlen $|M| \pm 1$ bereits seinen Raum an
Entschlüsselungsmöglichkeiten stark eingrenzen kann.


\section{Cäsar$^2$}

\subsection{}

\begin{itemize}
   \item Die Schlüsselpaare $(k_1,k_2)$ und $(k_2,k_1)$ führen zur selben
      Codierung, sodass die Schlüsselmenge effektiv wieder halbiert wird, da man
      nur eine von je zwei Möglichkeiten testen muss
   \item Es gilt
      \begin{IEEEeqnarray*}{rCl}
         E_{k_1,k_2}(M) & = & E_{k_2}(E_{k_1}(M)) \\
                        & = & [[M + k_1] + k_2]_{26} \\
                        & = & [M + k_1]_{26} \oplus_{26} [k_2]_{26} \\
                        & = & [M + k_1]_{26} \oplus_{26} k_2 \\
                        & = & [M]_{26} \oplus [k_1]_{26} \oplus_{26} k_2 \\
                        & = & [M]_{26} \oplus k_1 \oplus_{26} k_2 \\
                        & = & [M + (k_1 + k_2)]_{26}
      \end{IEEEeqnarray*}

      Somit stellt sich das selbe Problem wie bei der normalen
      Cäsar-Verschlüsselung. Sobald ein Klartextzeichen und das zugehörige
      Kryptozeichen bekannt sind, kann die Summe $(k_1 + k_2)$ berechnet werden.
      Zwar kennt man die Einzelschlüsseln nicht, kann mit deren Summe aber
      dennoch alle Texte entschlüsseln.
\end{itemize}

\subsection{}

Ein Gegenbeispiel für $k=2$ ist $E_2(13) = [2\cdot 13]_{26} = 0 = E_2(0)$. Ein
Gegenbeispiel für $k=13$ ist $E_{13}(2) = [13\cdot 2]_{26} = 0 = E_{13}(0)$.
Generell tritt das Problem dann auf, wenn der Schlüssel die
Klartextalphabetgröße ganzzahlig teilt. Dies ist hier nur für $k=2$ und $k=13$
der Fall.

\end{document}
