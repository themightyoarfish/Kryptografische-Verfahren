\documentclass{../crypto}
\sheet{2}
\date{30. Oktober 2015}

\begin{document}

\maketitle

\section{Multiplikation von Resten}

\subsection{}

\textcolor{orange}{Gäbe es ein inverses Element bezüglich $\odot_n$, so könnte
man Folgendes rechnen:}

% \vskip\baselineskip

\begin{IEEEeqnarray*}{rCl}
   [km]_n & = & [km^\prime] \\  
   k \odot_n m & = & k \odot_n m^\prime \\
   {}[k]_n \odot_n [m]_n & = & [k]_n \odot_n [m^\prime]_n \\
   k \odot_n m & = & k \odot_n m^\prime \\
   m & = & m^\prime
\end{IEEEeqnarray*}

\textcolor{orange}{Gibs aber nicht. Also habe ich einen ganzen Tag ohne
Fortschritt mit der Suche nach einer anderen Begründung verbracht. Jetzt ist es
9 Uhr. \textcolor{blue}{\#fuckmylife}}

\subsection{}

Für die Injektivität ist Folgendes zu zeigen.
\begin{equation*}
   E_k(m_1) = E_k(m_2) \Rightarrow m_1 = m_2
\end{equation*}

\begin{IEEEeqnarray*}{rCl}
   [m_1k + l]_n & = & [m_2k + l]_n \\{}
   [m_1k]_n \oplus_n [l]_n & = & [m_2k]_n \oplus_n [l]_n \\{}
   [m_1k]_n \oplus_n l \oplus_n (n - l) & = & [m_2k]_n \oplus_n l \oplus_n (n - l)\\{}
   [m_1k]_n & = & [m_2k]_n \\
\end{IEEEeqnarray*}

Nach Teilaufgabe a) folgt hieraus $m_1 = m_2$.
\end{document}
