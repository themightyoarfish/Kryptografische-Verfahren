\documentclass{../crypto}
\sheet{2}
\date{30. Oktober 2015}

\begin{document}

\maketitle

\section{Multiplikation von Resten}

\subsection{}

\begin{align*}
   \left[km\right]_n - \left[km^\prime\right]_n & = 0 \\
   k\odot_n m - k \odot_n m^\prime            & = 0 \\
   \left[km - km^\prime\right]_n              & = 0 \\
   \left[k\cdot(m - m^\prime)\right]_n        & = 0 \\
   k \odot_n (m - m^\prime)                   & = 0
\end{align*}

Das bedutet, dass $n$ $k \odot_n (m - m^\prime)$ teilt. Nach Vorraussetzung
sind $n$ und $k$ teilerfremd, sodass $n$ stattdessen $(m - m^\prime)$ teilen
muss. Da $m,m^\prime \in \mathbb{Z}_n$, kann $\left[m - m^\prime\right]_n = 0$ nur
gelten, falls $m = m^\prime$.

\hfill$\Box$


\subsection{}

Für die Injektivität ist Folgendes zu zeigen.
\begin{equation*}
   E_k(m_1) = E_k(m_2) \Rightarrow m_1 = m_2
\end{equation*}

\begin{IEEEeqnarray*}{rCl}
   \left[m_1k + l\right]_n                         & = & \left[m_2k + l\right]_n \\
   \left[m_1k\right]_n \oplus_n \left[l\right]_n   & = & \left[m_2k\right]_n \oplus_n \left[l\right]_n \\
   \left[m_1k\right]_n \oplus_n l \oplus_n (n - l) & = & \left[m_2k\right]_n \oplus_n l \oplus_n (n - l)\\
   \left[m_1k\right]_n                             & = & \left[m_2k\right]_n \\
\end{IEEEeqnarray*}

Nach Teilaufgabe a) folgt hieraus $m_1 = m_2$.
\hfill$\Box$

\section{Die Wahrscheinlichkeit von Kryptotexten in perfekt sicheren Kryptosystemen}

Sei $Y$ die Zufallsvariable über den möglichen Kryptotexten $y\in\mathcal{C}$.
Sei mit $\mathcal{C}(k) = \{c\in \mathcal{C} \mid \exists x\in \mathcal{M} :
E_k(x) = c\}$ die Menge der durch einen Schlüssel $k$ in Abhängigkeit
des Klartextes erzeugbaren Kryptotexte bezeichnet.
Für die Wahrscheinlichkeit, einen bestimmten Kryptotext zu erhalten, gilt 
\begin{align*}
   P(Y=y)             & = \sum\limits_{\{k \mid y\in\mathcal{C}(k)\}} P(K=k, E_k(x) = y) \\
   P(Y=y)             & = \sum\limits_{\{k \mid y\in\mathcal{C}(k)\}} P(K=k, X = E_k^{-1}(y)) \\
                      & = \sum\limits_{\{k \mid y\in\mathcal{C}(k)\}} P(K=k) ~ P(X = E_k^{-1}(y)) & \text{Schlüssel und Klartext unabhängig}\\
                      & = \frac{1}{|\mathcal{K}|}\sum\limits_{\{k \mid y\in\mathcal{C}(k)\}}
   P(X = E_k^{-1}(y)) & \text{Schlüssel gleichverteilt}\\
   \intertext{Da das System perfekt sicher ist, $E_k$ injektiv sowie
   $|\mathcal{C}|=|\mathcal{M}|$, gibt es für jedes
$(x,y)\in\mathcal{M}\times\mathcal{C}$ genau ein $k$ mit $E_k(x)=y$.} \\
                      &= \frac{1}{|\mathcal{K}|}\sum\limits_{\{k \mid y\in\mathcal{C}(k)\}} \frac{1}{|\mathcal{K}|} \\
   \intertext{Da $E_k$ injektiv ist, muss jedes $k$ auf
      $|\mathcal{M}|=|\mathcal{C}|$ verschiedene Kryptotexte abbilden. Dies
      bedeutet aber auch, dass jeder Kryptotext $c\in\mathcal{C}$ durch jedes
      $k\in\mathcal{K}$ generiert werden kann. Die obige Summe iteriert folglich
      über komplett $\mathcal{K}$, somit
   }
               P(Y=y) &= \frac{1}{|\mathcal{K}|}\cdot 1 \\
\end{align*}
\hfill$\Box$

\section{Eine Weitere Formel für die a-posteriori-Wahrscheinlichkeit}

Die Wahrscheinlichkeit des Ereignisses, einen bestimmten Kryptotext zu erhalten,
ergibt sich aus der Wahrscheinlichkeit, irgendeinen Klartext und gleichzeitig
einen ihn in den gegebenen Kryptotext überführenden Schlüssel zu erhalten. Es
kann mehrere solcher Kombinationen geben.
\begin{align*}
   P(C=c) &= \sum\limits_{m\in\mathcal{M}} \sum\limits_{k\in\mathcal{K}_{m,c}}P(M=m \cap K=k) \\
          &= \sum\limits_{m\in\mathcal{M}} \sum\limits_{k\in\mathcal{K}_{m,c}}P(M=m)\cdot(K=k) & \text{Schlüssel \& Klartext unabhängig} \\
          &= \sum\limits_{m\in\mathcal{M}} P(M=m) \sum\limits_{k\in\mathcal{K}_{m,c}}(K=k) \\
          &= \sum\limits_{m\in\mathcal{M}} P(M=m)\cdot P(\mathcal{K}_{m,c})
\end{align*}

Damit gilt 
\begin{align*}
   P(M=m \mid C=c) &= \frac{P(M=m \cap C=c)}{P(C=c)} \\
                   &= \frac{P(M=m)\cdot
P(\mathcal{K}_{m,c})}{\sum_{m'\in\mathcal{M}} P(M=m')\cdot
P(\mathcal{K}_{m',c})} & \text{Siehe VL, Folie 5.33}
\end{align*}

\hfill$\Box$

\section{Die Affine Verschlüsselung}

\subsection{}

\begin{align*}
E_{(k,l)}(m)&=\left[km+l\right]_{26}\\
c&=\left[km+l\right]_{26}\\
c&=\left[\left[km\right]_{26} + l\right]_{26}\\
\left[c - l\right]_{26}&=\left[km\right]_{26}
\end{align*}
Da $l \in \mathbb{Z}_{26}$ können wir $l$ als eine arbiträre Verschiebung ansehen
und somit auslassen, d.h. wenn wir eine Lösung für
$\left[c\right]_{26}=\left[km\right]_{26}$ finden, gibt es auch genau eine
passende Lösung für $\left[c - l\right]_{26}=\left[km\right]_{26}$.

Das bedeutet auch, dass $c$ und $km$ kongruent modulo $n$ sind.
% siehe
% https://de.wikipedia.org/wiki/Kongruenz_(Zahlentheorie)#L.C3.B6sbarkeit_von_linearen_Kongruenzen
Das wiederum bedeutet, dass $\left[c\right]_{26}=\left[km\right]_{26}$ genau
dann lösbar ist, wenn $ggT(k,n)$ $c$ teilt. Weil $ggT(k,n)=1$ gegeben ist, und
$\frac{c}{1}=c$ ist, ist somit die Gleichung lösbar -- genauer noch: Sie hat
genau $ggT(k,n)$ viele Lösungen.

Sie ist damit eindeutig lösbar in $m$, was im Umkehrschluss aber auch bedeutet,
dass es für jedes Paar $(m,c)$ genau eine Lösung von
$\left[c\right]_{26}=\left[km\right]_{26}$ gibt, womit gezeigt ist, dass die
Anzahl Schlüssel für jedes Paar konstant 1 ist.

\subsection{}

Wenn die Schlüssel gleichverteilt gewählt werden, ist die Wahrscheinlichkeit pro Schlüsselpaar $(k,l)$
\begin{align*}
   P(K=(k,l)) = \frac{1}{12\cdot 26} = \frac{1}{312}
\end{align*}
Wir können in die Formel der vorhergehenden Aufgabe einsetzen
\begin{align*}
P(M=m \mid C=c) &= \frac{P(M=m)\cdot P(\mathcal{K}_{m,c})}{\sum_{m'\in\mathcal{M}} P(M=m')\cdot P(\mathcal{K}_{m',c})} \\
            &= \frac{P(M=m)\cdot \frac{1}{312}}{\sum_{m'\in\mathcal{M}}
P(M=m')\cdot \frac{1}{312}} & \text{Schlüssel nach a) eindeutig}\\
            &= \frac{P(M=m)\cdot \frac{1}{312}}{\frac{1}{312}\cdot\sum_{m'\in\mathcal{M}} P(M=m')} \\
            &= \frac{P(M=m)}{\sum_{m'\in\mathcal{M}} P(M=m')} \\
            &= \frac{P(M=m)}{1} \\
            &= P(M=m)
\end{align*}
\hfill$\Box$ \\
Das heißt bei einer Gleichverteilung der Schlüssel ist AFFIN tatsächlich perfekt sicher.

\end{document}
