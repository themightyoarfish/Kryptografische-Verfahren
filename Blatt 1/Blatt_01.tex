\documentclass{../crypto}
\sheet{1}
\date{22. Oktober 2015}

\begin{document}

\maketitle


\section{Rechnen mit Resten}

\subsection{}

\begin{IEEEeqnarray*}{rCl}
   [a\cdot b]_n  & \stackrel{?}{=} & [a]_n \odot_n [b]_n \\
                 & =               & [(n\cdot q + r) \cdot (n\cdot p + s)]_n \\
                 & =               & [n^2pq + nqs + npr + rs]_n \\
                 & =               & [n^2pq]_n \oplus_n [nqs]_n \oplus_n [npr]_n \oplus_n [rs]_n \\
                 & =               & [rs]_n \\
                 & =               & r \odot_n s \\
                 & =               & [a]_n \odot_n [b]_n
\end{IEEEeqnarray*}

\subsection{}

\begin{IEEEeqnarray*}{rCl}
   \left(\left(3126 \oplus_{11} 21783\right) \odot_{11} 213894\right)
   \oplus_{11} 31942 & = & \left(\left[24909\right]_{11} \odot_{11} 213894\right) \oplus_{11} 31942 \\
                     & = & \left(5 \odot_{11} 213894\right) \oplus_{11} 31942 \\
                     & = & \left[1069470\right]_{11} \oplus_{11} 31942 \\
                     & = & 6 \oplus_{11} 31942 \\
                     & = & \left[31948\right]_{11} \\
                     & = & 4
\end{IEEEeqnarray*}

\subsection{}

\begin{IEEEeqnarray*}{rCltt}
   \left[2131897^8\right]_3   & = & \left[\prod_{i=1}^8 2131897\right]_3\\
                              & = & \prod_{i=1}^8 \left[2131897\right]_3   & \hspace{1cm} & Aufgabe 1.1 a) \\
                              & = & \prod_{i=1}^8 1             & \hspace{1cm} &
   \begin{IEEEeqnarraybox}[][c]{s}
      Es sind $21\cdot x$, $3\cdot x$, $18\cdot x$ \\
      und $9\cdot x$ durch $3$ teilbar, \\
      sodass nur $7$ für Rest sorgen kann\\
   \end{IEEEeqnarraybox} \\
   & = & 1
\end{IEEEeqnarray*}


\section{Die Skytale}

\subsection{}

\subsection{}

\subsection{}


\section{Cäsar$^2$}

\subsection{}

\begin{itemize}
   \item Die Schlüsselpaare $(k_1,k_2)$ und $(k_2,k_1)$ führen zur selben
      Codierung, sodass die Schlüsselmenge effektiv wieder halbiert wird, da man
      nur eine von je zwei Möglichkeiten testen muss
   \item Es gilt
      \begin{IEEEeqnarray*}{rCl}
         E_{k_1,k_2}(M) & = & E_{k_2}(E_{k_1}(M)) \\
                        & = & [[M + k_1] + k_2]_{26} \\
                        & = & [M + k_1]_{26} \oplus_{26} [k_2]_{26} \\
                        & = & [M + k_1]_{26} \oplus_{26} k_2 \\
                        & = & [M]_{26} \oplus [k_1]_{26} \oplus_{26} k_2 \\
                        & = & [M]_{26} \oplus k_1 \oplus_{26} k_2 \\
                        & = & [M + (k_1 + k_2)]_{26}
      \end{IEEEeqnarray*}

      Somit stellt sich das selbe Problem wie bei der normalen
      Cäsar-Verschlüsselung. Sobald ein Klartextzeichen und das zugehörige
      Kryptozeichen bekannt sind, kann die Summe $(k_1 + k_2)$ berechnet werden.
      Zwar kennt man die Einzelschlüsseln nicht, kann mit deren Summe aber
      dennoch alle Texte entschlüsseln.
\end{itemize}

\subsection{}

\end{document}
