\documentclass{../crypto}
\sheet{5}
\date{4. Dezember 2015}

\begin{document}

\maketitle

\section{Der Geburtstagsangriff mit konstantem Speicher terminiert}

Es sei mit $P(n,k)$ die Wahrscheinlichkeit bezeichnet, bei $n$ Versuchen ($n$
verschiedenen Eingaben) bei einer Hashlänge von $2^k$ Bits keine Kollision zu
erhalten.

\begin{align*}
   P(n,k)  & = 1 \cdot \frac{2^k - 1}{2^k} \cdot \ldots \cdot \frac{2^k - n + 1}{2^k} \\
   P(n,k)  & =   \prod_{i=0}^{n-1} \frac{2^k - i}{2^k}                                \\
           & =   \prod_{i=1}^{n-1} (1 -  \frac{i}{2^k})                               \\
\intertext{Da $(1-x)$ durch $e^{-x}$ nach oben abgeschätzt werden kann, gilt}
           & \le \prod_{i=1}^{n-1} e^{-\frac{i}{2^k}} \\
           & =   e^{\sum_{i=1}^{n-1}-\frac{i}{2^k}}   \\
           & =   e^{-\frac{1}{2^k}\sum_{i=1}^{n-1}i}  \\
           & =   e^{-\frac{1}{2^k} \frac{n(n-1)}{2}}  \\
           & \le   e^{-\frac{(n-1)^2}{2\cdot 2^k}}    \\
\end{align*}

Gesucht ist nun $n$, sodass $P(n,128) = \frac{1}{4}$.

\begin{align*}
   e^{-\frac{(n-1)^2}{2\cdot 2^k}} & = \frac{1}{4}                        \\
   -\frac{(n-1)^2}{2\cdot 2^k}     & = \ln{\frac{1}{4}}                   \\
   -(n-1)^2                        & = 2\cdot 2^k \cdot \ln{\frac{1}{4}}  \\
   -n^2 + 2n - 1                   & = 2\cdot 2^k \cdot \ln{\frac{1}{4}}  \\
   n^2 - 2n + 1                    & = -2\cdot 2^k \cdot \ln{\frac{1}{4}} \\
   \intertext{Mit den üblichen Verfahren, z.B. $pq$-Formel und
   Computerunterstützung lässt dies als positive Lösung zu}
   n_{128}                         & \approx 30.715.843.678.825.642.450 \ge 3.0716 \cdot 10^{19}
\end{align*}

Man müsste also mehr also über dreißig Trillionen
Versuche machen. Analog gelangt man für $k=160$ zu dem Ergebnis 
\begin{equation*}
   n_{160} \approx 2.012.993.531.335.517.303.552.701 \ge 2.013 \cdot 10^{24}
\end{equation*}
oder etwa $2$ Quadrillionen Versuche.

\section{Der Geburtstagsangriff mit konstantem Speicher findet Kollisionen}

Falls es $1 \le 1 < I < J$ mit $x_I = x_J$ und damit $H(x_{I-1}) = H(x_{J-1})$
gibt, so hat die Folge $x_1,\ldots,x_q$ offenbar eine Periode von $J - I$. Der
$(J+i)$-te Wert ist also gleich dem $(I+i)$-ten. Falls man die Periodizität schon für $i
< I$ annehmen kann, gilt $x_{J-I} = x_{J-I + J-I} = x_{2(J-I)}$.

Allgemein stimmt die Aussage aber nicht. Gilt zum Beispiel $x_7 = x_{12}$, so
ist $x_8 = x_{13}, x_9 = x_{14},\ldots,x_{10} = x_{15}$.


\section{Schlüsseltauschprotokolle}
\paragraph{a)} Lässt man den Zeitstempel beim Breitmaulfroschprotokoll 
\paragraph{b)} 
\paragraph{c)} Da Alice A und den Sessionkey weiß, kann sie aus E_kB(A,k_B) und A und k_B auf k_B schließen
\section{Ein sicheres Protokoll?}
\end{document}
