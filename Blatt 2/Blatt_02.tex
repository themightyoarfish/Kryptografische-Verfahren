\documentclass{../crypto}
\sheet{2}
\date{30. Oktober 2015}

\begin{document}

\maketitle

\section{Multiplikation von Resten}

\subsection{}

\textcolor{orange}{Gäbe es ein inverses Element bezüglich $\odot_n$, so könnte
man Folgendes rechnen:}

% \vskip\baselineskip

\begin{IEEEeqnarray*}{rCl}
   \left[km\right]_n                         & = & \left[km^\prime\right] \\
   k \odot_n m                               & = & k \odot_n m^\prime \\
   % \left[k\right]_n \odot_n \left[m\right]_n & = & \left[k\right]_n \odot_n \left[m^\prime\right]_n \\
   % k \odot_n m                               & = & k \odot_n m^\prime \\
   m                                         & = & m^\prime
\end{IEEEeqnarray*}

\textcolor{orange}{Gibs aber nicht. $\mathbb{Z}_n$ mit $\odot_n$ ist keine
   Gruppe, weil $0$ kein Inverses haben kann ($0\cdot ? = 1$). Also habe ich einen ganzen Tag ohne
Fortschritt mit der Suche nach einer anderen Begründung verbracht. Jetzt ist es
9 Uhr. \textcolor{blue}{\#fuckmylife}}

Noch ne Idee:
\begin{align*}
   \left[km\right]_n - \left[km^\prime\right] & = 0 \\
   k\odot_n m - k \odot_n m^\prime            & = 0 \\
   \left[km - km^\prime\right]_n              & = 0 \\
   \left[k\cdot(m - m^\prime)\right]_n        & = 0 \\
   k \odot_n (m - m^\prime)                   & = 0
\end{align*}

Das bedutet, dass $n$ $k \odot_n (m - m^\prime)$ teilt. Nach Vorraussetzung
sind $n$ und $k$ teilerfremd, sodass $n$ stattdessen $(m - m^\prime)$ teilen
muss. Da $m,m^\prime \in \mathbb{Z}_n$, kann $\left[m - m^\prime\right] = 0$ nur
gelten, falls $m = m^\prime$.
\subsection{}

Für die Injektivität ist Folgendes zu zeigen.
\begin{equation*}
   E_k(m_1) = E_k(m_2) \Rightarrow m_1 = m_2
\end{equation*}

\begin{IEEEeqnarray*}{rCl}
   \left[m_1k + l\right]_n                         & = & \left[m_2k + l\right]_n \\
   \left[m_1k\right]_n \oplus_n \left[l\right]_n   & = & \left[m_2k\right]_n \oplus_n \left[l\right]_n \\
   \left[m_1k\right]_n \oplus_n l \oplus_n (n - l) & = & \left[m_2k\right]_n \oplus_n l \oplus_n (n - l)\\
   \left[m_1k\right]_n                             & = & \left[m_2k\right]_n \\
\end{IEEEeqnarray*}

Nach Teilaufgabe a) folgt hieraus $m_1 = m_2$.
\end{document}
