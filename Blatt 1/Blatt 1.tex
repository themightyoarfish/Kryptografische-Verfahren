\documentclass{../Crypto}
\sheet{1}
\begin{document}
   \maketitle

   \section{Rechnen mit Resten}

   \subsection{}

   \begin{IEEEeqnarray*}{rCl}
      [a\cdot b]_n  & \stackrel{?}{=} & [a]_n \odot_n [b]_n \\
                    & =               & [(n\cdot q + r) \cdot (n\cdot p + s)]_n \\
                    & =               & [n^2pq + nqs + npr + rs]_n \\
                    & =               & [n^2pq]_n \oplus_n [nqs]_n \oplus_n [npr]_n \oplus_n [rs]_n \\
                    & =               & [rs]_n \\
                    & =               & r \odot_n s \\
                    & =               & [a]_n \odot_n [b]_n
   \end{IEEEeqnarray*}

   \subsection{}
   
   \subsection{}
   
   \begin{IEEEeqnarray*}{rCltt}
      [2131897^8]_3   & = & [\prod_{i=1}^8 2131897]_3\\
                      & = & \prod_{i=1}^8 [2131897]_3   & \hspace{1cm} & Aufgabe 1.1 a) \\
                      & = & \prod_{i=1}^8 1             & \hspace{1cm} &
      \begin{IEEEeqnarraybox}[][c]{s}
         Es sind $21\cdot x$, $3\cdot x$, $18\cdot x$ \\
         und $9\cdot x$ durch $3$ teilbar, \\
         sodass nur $7$ für Rest sorgen kann\\
      \end{IEEEeqnarraybox} \\
                     & = & 1
   \end{IEEEeqnarray*}

   \end{document}
