\documentclass{../crypto}
\usepackage{amsthm} 
\usepackage{mathtools}
\sheet{6}
\date{11. Dezember 2015}

\newtheorem{theorem}{Theorem}
\usepackage{amssymb}

\begin{document}

\maketitle

\section{Exponentiation von Hand}

\subsection{}
\begin{align*}
  \left[3^{1000}\right]_{100}                             & = \left[3^{512+256+128+64+32+8}\right]_{100}               \\
                                                          & = \left[\left[\left[\left[\left[\left[3^{512}\right]_{100}
                              \cdot 3^{256}\right]_{100}
                              \cdot 3^{128}\right]_{100}
                              \cdot 3^{64}\right]_{100}
                              \cdot 3^{32}\right]_{100}
                              \cdot 3^{8}\right]_{100}                                                                 \\
                                                          & = \left[\left[\left[\left[\left[41 \cdot 21\right]_{100}
                              \cdot 61\right]_{100}
                              \cdot 81\right]_{100}
                              \cdot 41\right]_{100}
                              \cdot 61\right]_{100}                                                                    \\
                                                          & = \left[\left[\left[\left[61 \cdot 61\right]_{100}
                              \cdot 81\right]_{100}
                              \cdot 41\right]_{100}
                              \cdot 61\right]_{100}                                                                    \\
                                                          & = \left[\left[\left[21 \cdot 81\right]_{100}
                              \cdot 41\right]_{100}
                              \cdot 61\right]_{100}                                                                    \\
                                                          & = \left[\left[1 \cdot 41\right]_{100}
                              \cdot 61\right]_{100}                                                                    \\
                                                          & = \left[41 \cdot 61\right]_{100}                           \\
                                                          & = 1
\end{align*}

\textbf{Nebenrechnungen:}
\begin{align*}
  \left[3^{1}\right]_{100}                                                            & = 3                            \\
  \left[3^{2}\right]_{100}                                                            & = 9                            \\
  \left[3^{4}\right]_{100}                                                            & = \left[3^{2+2}\right]_{100}
                            = \left[\left[3^{2}\right]_{100} \cdot 3^{2}\right]_{100}
                            = \left[9 \cdot 3^{2}\right]_{100} = 81                                                    \\
  \left[3^{8}\right]_{100}                                                            & = \left[3^{4+4}\right]_{100}
                            = \left[81 \cdot 3^{4}\right]_{100}
                            = \left[6561\right]_{100} = 61                                                             \\
  \left[3^{16}\right]_{100}                                                           & = \left[3^{8+8}\right]_{100}
                             = \left[3721\right]_{100} = 21                                                            \\
  \left[3^{32}\right]_{100}                                                           & = \left[441\right]_{100} = 41  \\
  \left[3^{64}\right]_{100}                                                           & = \left[1681\right]_{100} = 81 \\
  \left[3^{128}\right]_{100}                                                          & = 61                           \\
  \left[3^{256}\right]_{100}                                                          & = 21                           \\
  \left[3^{512}\right]_{100}                                                          & = 41                           \\
  \left[3^{1024}\right]_{100}                                                         & = 81                           \\
                                                                                      & \vdots
\end{align*}
Wir beobachten die Periode $81, 61, 21, 41, \dots$, die wir uns zu Nutze machen können.

\subsection{}
\begin{align*}
  \left[101^{4800000002}\right]_{35} & = \left[ \left[16^6\right]_{35} \cdot \left[11^5\right]_{35} \right]_{35} \\
                                     & = \left[11^5\right]_{35}                                                  \\
                                     & = 16
\end{align*}

\textbf{Nebenrechnungen:}
\begin{align*}
  \left[101^{1}\right]_{35} &= 31  \\
  \left[101^{2}\right]_{35} &= 16  \\
  \left[101^{4}\right]_{35} &= 11  \\
  \left[101^{8}\right]_{35} &= 16  \\
  \left[101^{16}\right]_{35} &= 11 \\
                             &\vdots
\end{align*}
Wir beobachten, dass jede zweite dieser Potenzen (also $2, 8, 32, \dots$) einen
Rest von $16$ hat, alle anderen (mit Ausnahme von $1$) den Rest von $11$.
Mit diesem Wissen können wir mit der Binärdarstellung und einfachem Zählen
herausfinden, wie oft wir $11$ bzw. $16$ multiplizieren müssen, um auf das 
Ergebnis zu kommen:

\begin{align*}
   4800000002_{10} = 1 00 01 11 10 00 01 10 10 00 11 00 00 00 00 00 10_2
\end{align*}

Wir finden sechs Potenzen, deren Rest $16$ ist und fünf Potenzen deren Rest $11$ 
ist. Es folgt:

\begin{align*}
  \left[101^{4800000002}\right]_{35} &= \left[ \left[16^6\right]_{35} \cdot \left[11^5\right]_{35} \right]_{35}
\end{align*}

Wir können nun zunächst $\left[16^6\right]_{35}$ geschickt berechnen:

\begin{align*}
  \left[16^6\right]_{35}  & = \left[16^{2+2+2}\right]_{35}           \\
                          & = \left[11 \cdot 11 \cdot 11\right]_{35} \\
                          & = 1                                      \\
\end{align*}

Wir stellen fest, dass wir diese Erkenntnis wiederum in unsere Gleichung 
einsetzen können und erhalten:

\begin{align*}
  \left[ \left[16^6\right]_{35} \cdot \left[11^5\right]_{35} \right]_{35} & = \left[1 \cdot \left[11^5\right]_{35} \right]_{35} \\
                                                                          & = \left[11^5\right]_{35}
\end{align*}

Das können wir erneut geschickt berechnen:
\begin{align*}
  \left[11^{5}\right]_{35}  & = \left[11^{4+1}\right]_{35}    \\
                            & = \left[11 \cdot 11\right]_{35} \\
                            & = 16
\end{align*}
  
\emph{Nebenrechnung:}
\begin{align*}
  \left[11^{1}\right]_{35} &= 11 \\
  \left[11^{2}\right]_{35} &= 16 \\
  \left[11^{4}\right]_{35} &= 11
\end{align*}


\section{Erweiterter Euklidischer Algorithmus}

\subsection{}

\begin{align*}
  a          & = 23                                                \\
  b          & = 17                                                \\
  [a]_b      & \stackrel{?}{\neq} 0 \quad \checkmark               \\
  r          & =                 [23]_{17} = 6 \neq 0              \\
  q          & =                 \lfloor \frac{23}{17} \rfloor = 1 \\
  (d, x, y)  & =                 \textit{$eEuklid$}(17, 6)         \\
             & =                 (1, -1, 3)                        \\
             & \Rightarrow       (d, y, x - qy) = (1, 3, -4)
\end{align*}

% Müssten wir nicht eigentlich $(d, x, y) = \textit{$eEuklid$}(23, 17)$ berechnen?
% Hab ich hier mal gemacht:

%$\textit{$eEuklid$}(23, 17)$:
%\begin{align*}
  %d &= ggT(23, 17) = 1 \\
  %\Rightarrow 1 &= 23x + 17y          
%\end{align*}
%
%Bestimmen von $x$ und $y$:
%
%\begin{center}
  %\begin{tabular}{l|c|c|l}
    %Iteration & $x$ & $y$ & $|23x - 17y| - 1$ \\\hline
            %1 &  1  &  1  & $|23 - 17| - 1 =   5$ \\
            %2 &  1  &  2  & $|23 - 34| - 1 =   8$ \\
            %3 &  2  &  2  & $|46 - 34| - 1 =  11$ \\
            %4 &  2  &  3  & $|46 - 51| - 1 =   4$ \\
            %5 &  3  &  3  & $|69 - 51| - 1 =  17$ \\
            %6 &  3  &  4  & $|69 - 68| - 1 =   0$ \\
  %\end{tabular}
%\end{center}
%
%Daraus folgt $x = 3$ und $y = -4$, wir überprüfen: 
%$23 \cdot 3 - 17 \cdot 4 = 1$. Das bedeutet unser Ergebnis ist 
%$\textit{$eEuklid$}(23, 17) = (1, 3, -4)$.

\subsection{}

Wir zeigen zunächst Folgendes:

\begin{theorem}
Falls $a = \floor*{\frac{a}{b}}b + [a]_b = qb + r$, dann
   ist $ggt(a,b) = ggt(b,r)$
\end{theorem}
\begin{proof}
   Sei $d=ggT(a,b)$, also gilt
   \begin{align*}
      a & = dn \\
      b & = dm \\
      \intertext{Und daher}
      a - b & = dn - dm  = d(n-m) = dl
   \end{align*}
   Also is $d$ Teiler von $a$ sowie von $a-b$. Alle gemeinsamen Teiler von $a$
   und $b$ sind also Teiler von $a-b$. Deshalb gilt
   \begin{align*}
      b & = a - (a - b) = dn - dl = d(n - l) = dk
   \end{align*}
   und alle gemeinsamen Teiler von $a$ und $a-b$ sind Teiler von $b$. Die
   wiederholte Subtraktion von $b$ von $a$ ändert also nichts an den gemeinsamen
   Teilern. Es folgt $ggT(a,b) = ggT(b,a) = ggT(b, a - b) = ggT(b, a - b - b) =
   \ldots = ggT(b, a - qb) = ggT(b,r)$.
\end{proof}

Es folgt also aus der zu zeigenden Aussage
\begin{align*}
   ggT(b,r) = xb + yr & = ggT(a,b) = ya + xb - qyb     \\
            yr        & = ya - yqb                     \\
            y [a]_b   & = ya - yqb                     \\
                      & = ya -y \floor*{\frac{a}{b}} b \\
                      & = ya - y(a - [a]_b)            \\
                      & = ya - ya + y [a]_b            \\
                      & = y [a]_b
\end{align*}
Da aus der Aussage eine Tautologie folgt, muss sie stimmen.

\subsection{}

Die Korrektheit ist nach \textbf{b)} trivial. Im zweiten Schritt berechnet der
Algorithmus $(d,x,y) = eEuklid(b,r)$. Nach dem eben Gezeigten ist $d = d^\prime$
mit $(d^\prime, x^\prime,y^\prime) = eEuklid(b,r)$. Im dritten Schritt gibt der
Algorithmus neben dem $ggT$ die beiden Faktoren $y$ und $x-qy$ zurück. Nach
Theorem $1$ sind dies die richtigen Faktoren für $ggT(a,b)$.

\section{Merkle-Hellmann-Verfahren}

\subsection{}

Um $w^{-1}$ zu bestimmen, muss $eEuklid$ mit $385$ und $17$ durchgeführt werden. Das
Ergebnis ist dann $(d,x,y) = (1,-3,68)$. Wir überprüfen, ob es sich bei $68$
tatsächlich um das multiplikativ Inverse von $17$ in $\mathbb{Z}_{385}^*$ handelt.

\begin{align*}
   [17 \cdot 68]_{385} = [1156]_{385} = 1
\end{align*}

\subsection{}

Als öffentlichen Schlüssel haben wir gegeben 
$a = (17, 23, 46, 51, 92, 102, 184, 204)$ und $n = 385$.

Wir können so die Nachricht von Bob einfach verschlüsseln:

\begin{align*}
  \textit{Enc}_{a,n}(m) & = [<a, m>]_n                                                              \\
                        & = [<(17, 23, 46, 51, 92, 102, 184, 204), (1, 0, 1, 1, 1, 0, 1, 1)>]_{385} \\
                        & = [17 + 46 + 51 + 92 + 184 + 204]_{385}                                   \\
                        & = [594]_{385}                                                             \\
                        & = 209
\end{align*}

\subsection{}

Um die Nachricht zu entschlüsseln, müssen wir das folgende 
\textsc{SubsetSum}-Problem lösen:
\begin{align*}
  \textit{Dec}_{a,n}(c)  & = \text{\textsc{SubsetSum}}\left(b, [c \cdot w^{-1}]_{n}\right) \\
                         & = \text{\textsc{SubsetSum}}\left(b, [17 \cdot 68]_{385}\right)  \\
                         & = \text{\textsc{SubsetSum}}\left(b, [1156]_{385}\right)         \\
                         & = \text{\textsc{SubsetSum}}\left(b, 208\right)                  \\
\end{align*}
Die fehlende Komponente $b$ können wir berechnen:
\begin{align*}
  b  & = [w^{-1} \cdot a]_{n}                                                  \\
     & = \left[68 (17, 23, 46, 51, 92, 102, 184, 204)\right]_{385}             \\
     & = \left[(1156, 1564, 3128, 3468, 6256, 6936, 12512, 13872)\right]_{385} \\
     & = (1, 24, 48, 3, 96, 6, 192, 12)
\end{align*}

Jetzt lösen wir das Problem:
\begin{align*}
  \textit{Dec}_{a, n}(c)  & = \text{\textsc{SubsetSum}}\left((1, 24, 48, 3, 96, 6, 192, 12), 208\right) \\
                          & = (1, 0, 0, 1, 0, 0, 1, 1)
\end{align*}

Damit ist die Lösung $m = 10010011$.

\end{document}
