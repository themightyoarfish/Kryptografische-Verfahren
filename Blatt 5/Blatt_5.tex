\documentclass{../crypto}
\sheet{5}
\date{4. Dezember 2015}

\usepackage{amssymb} 

\begin{document}

\maketitle

\section{Der Geburtstagsangriff mit konstantem Speicher terminiert}

Es sei mit $P(n,k)$ die Wahrscheinlichkeit bezeichnet, bei $n$ Versuchen ($n$
verschiedenen Eingaben) bei einer Hashlänge von $k$ Bits keine Kollision zu
erhalten.

\begin{align*}
   P(n,k)  & = 1 \cdot \frac{2^k - 1}{2^k} \cdot \ldots \cdot \frac{2^k - n + 1}{2^k} \\
   P(n,k)  & =   \prod_{i=0}^{n-1} \frac{2^k - i}{2^k}                                \\
           & =   \prod_{i=1}^{n-1} (1 -  \frac{i}{2^k})                               \\
\intertext{Da $(1-x)$ durch $e^{-x}$ nach oben abgeschätzt werden kann, gilt}
           & \le \prod_{i=1}^{n-1} e^{-\frac{i}{2^k}} \\
           & =   e^{\sum_{i=1}^{n-1}-\frac{i}{2^k}}   \\
           & =   e^{-\frac{1}{2^k}\sum_{i=1}^{n-1}i}  \\
           & =   e^{-\frac{1}{2^k} \frac{n(n-1)}{2}}  \\
           & \le   e^{-\frac{(n-1)^2}{2\cdot 2^k}}    \\
\end{align*}
Gesucht ist nun $n$, sodass $P(n,128) \le \frac{1}{4}$.
\begin{align*}
   e^{-\frac{(n-1)^2}{2\cdot 2^k}} & \le \frac{1}{4}                        \\
   -\frac{(n-1)^2}{2\cdot 2^k}     & \le \ln{\frac{1}{4}}                   \\
   -(n-1)^2                        & \le 2\cdot 2^k \cdot \ln{\frac{1}{4}}  \\
   -n^2 + 2n - 1                   & \le 2\cdot 2^k \cdot \ln{\frac{1}{4}}  \\
   n^2 - 2n + 1                    & \ge -2\cdot 2^k \cdot \ln{\frac{1}{4}} \\
   \intertext{Mit den üblichen Verfahren, z.B. $pq$-Formel und
   Computerunterstützung lässt dies als positive Lösung zu}
   n_{128}                         & \gtrapprox 30.715.843.678.825.642.450 \approx 3.0716 \cdot 10^{19}
\end{align*}

Man müsste also mehr also über dreißig Trillionen
Versuche machen. Analog gelangt man für $k=160$ zu dem Ergebnis 

\begin{equation*}
   n_{160} \gtrapprox 2.012.993.531.335.517.303.552.701 \approx 2.013 \cdot 10^{24}
\end{equation*}
oder etwa $2$ Quadrillionen Versuche.

\section{Der Geburtstagsangriff mit konstantem Speicher findet Kollisionen}

Falls es $1 \le I < J$ mit $x_I = x_J$ und damit $H(x_{I-1}) = H(x_{J-1})$
gibt, so hat die Folge $x_1,\ldots,x_q$ offenbar eine Periode von $J - I$. Der
$(J+i)$-te Wert ist also gleich dem $(I+i)$-ten. 

In einer Folge der Periode $J-I$ gibt es zwischen (inklusive) den ersten
gleichen Elementen eine Zahl, deren Index ein Vielfaches der Periode ist. Dies
ist an Beispielen leicht zu sehen und zur Not beweisbar, indem man ein Fenster
der Länge $J-I$ über $\mathbb{N}$ wandern lässt. Also
gilt, es existiert ein $x_{l(J-I)} = x_{l(J-I) + (J-I)} = x_{l(J-I) + 2(J-I)} =
\ldots = x_{l(J-I) + l(J-I)} = x_{2l(J-I)}$, wobei $l(J-I) < J$ ist.

\section{Schlüsseltauschprotokolle}
\subsection{}

Der Breitmaulfrosch ist nicht einmal gegen Angriffe ohne bekannten Schlüssel
sicher, wenn es keinen Zeitstempel gibt. Fängt der Angreifer eine Nachricht ab,
deren Inhalt er kennt (hierfür könnte er den bekannten Schlüssel vielleicht
verwenden), so könnte er diese erneut senden und erneut eine Reaktion auslösen
(z.B. eine Überweisung o.Ä.). 

Ein Problem ist auch, dass vor allem ohne Zeitstempel (aber potenziell auch mit,
wenn der Angriff innerhalb eines Zeitfensters durchgeführt wird, das nicht
verdächtig ist) ein Angreifer die Validität eines Schlüssels erhalten kann,
indem er abwechseln Alice und Bob vertritt und die Nachricht
$E_{K_{BC}}(t_Z, A, K_{AB})$ an Charlie sendet, die dieser zuvor an Bob
geschickt hatte. Charlie sendet dann $E_{K_{AC}}(t_Z^\prime, B, K_{AB})$ an
Alice und der Schlüssel $K_{AB}$ bleibt aktiv. Sendet man die Nachricht immer
schnell genug zurück an Charlie, kann der Schlüssel so am Leben erhalten werden.

% Lässt man den Zeitstempel beim Breitmaulfroschprotokoll weg, kann ein Angreifer
% durch Abfangen der Nachrichten um Charlie mit zwei verschiedenen Kryptoptexten
% und einem (mindestens) teilweise bekannten Klartext die Kommunikation abhören.

\subsection{}

Das Needham-Schroeder-Protokoll ist \textbf{nicht} sicher gegen Angriffe mit
bekanntem Sitzungsschlüssel mit Zusatzinformation. Angenommen, Everett kennt einen alten
Sitzungsschlüssel $k_{AB}$ und hat die dritte Nachricht $E_{k_{SB}}(A,k_{AB})$
(also die Nachricht, in der Charlie ein für Bob bestimmtes Chiffrat an Alice
sendet) aufgezeichnet. Er kann nun die Nachricht erneut an Bob senden, welcher
mit seiner verschlüsselten Zufallszahl $E_{k_{AB}}(r_B)$ antwortet. Diese kann
Everett entschlüsseln, ändern, verschlüsseln und wieder an Bob schicken, sodass
dieser dann denkt, er würde mit Alice kommunizieren.

Ist allerdings \textbf{nur} der Sitzungsschlüssel bekannt, kann ein Angreifer
damit nichts anfangen, denn die Nonce $r_A$ garantiert Alice dass der erhaltene
Schlüssel aktuell ist und die Schlüssel werden von Charlie unabhängig generiert.

Bei Otway-Rees existiert das Problem mit der fehlenden Authentizität nicht
(s.o.). Bob kann also nicht so einfach über die Identität seines
Kommunikationspartners getäuscht werden. Auch hier generiert Charlie unabhängig
die Sitzungsschlüssel $k_{AB}$ und Alice und Bob werden durch die Noncen $r_A$
und $r_B$ von der Aktualität überzeugt.

% Das Otway-Rees-Protokoll sicher gegen Angriffe
% mit bekannten Sitzungsschlüsseln, weil Charlie diesen immer neu generiert. Wir
% hoffen, dass Charlie ein Verfahren verwendet, dass es nicht zulässt von alten
% Schlüsseln auf neue Schlüssel zu schließen -- z.B.  unter Verwendung von
% Timestamps. Sind diese Timestamps nicht vorhanden, sind Replay-Attacken jedoch
% weiterhin möglich.

% Das Diffie-Hellman-Protokoll ist sicher gegen Angriffe mit bekannten 
% Sitzungsschlüsseln, weil für jeder Sitzung sehr trivial neue Schlüssel erzeugt
% werden können, denn es müssen nur Zufallszahlen $a$ und $b$ von Alice und Bob
% generiert sowie zwei Nachrichten ausgetauscht werden

Für den Diffie-Hellman-Schlüsseltausch werden in jeder Sitzung die Exponenten
$a$ und $b$ neu zufällig generiert. Wenn ein Angreifer einen Schlüssel $(g^a \mod
p)^b \mod p$ kennt, kann er nicht einfach $a$ oder $b$ berechnen oder auch bloß
$(g^a \mod p)$. Mit der letzen Information könnte er eine MITM-Attacke
durchführen, kennt er aber nur den alten Schlüssel, geht dies nicht.

\subsection{}

Da Alice $A$ und den Sitzungsschlüssel aus $E_{k_A}(r_A,k_{AB},B,E_{k_B}(A,k_{AB}))$
weiß, kann sie aus $E_{k_B}(A,k_{AB})$, $A$ und $k_{AB}$ auf $k_B$ schließen, je
nach Verschlüsselungsverfahren (z.B. Rainbow Tables). Der
Angriff funktioniert auch, wenn das Protokoll um einen Timestamp erweitert wird.

\section{Ein sicheres Protokoll?}
\subsection{}
\begin{align*}
	w \oplus t & = u \oplus r \oplus t                   \\
	           & = s \oplus t \oplus r \oplus t          \\
	           & = k \oplus r \oplus t \oplus r \oplus t \\
	           & = k \oplus r \oplus r \oplus t \oplus t \\
	           & = k \oplus 0 \oplus 0                   \\
	           & = k
\end{align*}

\subsection{}
\begin{enumerate}
    \item Erste Nachricht von Alice abfangen $\rightarrow s := k \oplus r$
    \item Erste Nachricht von Bob abfangen $\rightarrow u := s \oplus t$
    \item Zweite Nachricht von Alice abfangen $\rightarrow w := u \oplus r$
    \item Berechnen: 
        \begin{align*}
           r & = w \oplus u \quad \left(u \oplus r \oplus u\right)\\
           k & = s \oplus r \quad \left(k \oplus r \oplus r\right)
        \end{align*}
\end{enumerate}
Damit hat der Angreifer den Schlüssel, der von Alice und Bob verwendet wird.
\end{document}
