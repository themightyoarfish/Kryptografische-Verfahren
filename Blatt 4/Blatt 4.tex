\documentclass{../crypto}
\sheet{4}
\date{27. November 2015}

\usepackage{mathtools}  % coloneqq
\begin{document}

\maketitle

\section{Zufallsgeneratoren}

\subsection{}

Sei $G$ ein PZFG mit $|G(s)| > 4|s| = 4n$. Definiere $G^\prime \coloneqq
G(s_1,\ldots,s_{\lfloor \frac{n}{2} \rfloor})$. Nach Voraussetzungen ist
$G^\prime$ PZFG. Da $|G^\prime| > 2|s|$, gilt 
\begin{equation*}
   H(s) = G^\prime(0^{|s|} || s) = G(0^{|s|})
\end{equation*}
Also berechnet $H$ für alle $s$ den selben Wert und ist damit leicht von echtem
Zufall unterscheidbar. Ein Distinguisher $\mcal{D}$ muss nur für gegebenes $s$
den Wert $G(0^{|s|})$ berechnen und prüfen, ob $H(s) = G(0^{|s|})$. Daher ist
$P(\mcal{D}(G(s)) = 1) = 1$, während die Situation, dass ein zufälliger
Bistring $r$ gerade $r=G(0^{|s|})$ ist, nur eine von $2^{|s|}$ Möglichkeiten
ist, $r$ zu wählen, mithin $P(\mcal{D}(r) = 1) = \frac{1}{2^{|s|}}$. Die
Wahrscheinlichkeiten unterscheiden sich also nicht-vernachlässigbar.
\end{document}
