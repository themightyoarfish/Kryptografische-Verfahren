\documentclass{../crypto}
\sheet{1}
\date{22. Oktober 2015}

\begin{document}

\maketitle


\section{Rechnen mit Resten}

\subsection{}

    \begin{IEEEeqnarray*}{rCl}
        [a\cdot b]_n  & \stackrel{?}{=} & [a]_n \odot_n [b]_n \\
                      & =               & [(n\cdot q + r) \cdot (n\cdot p + s)]_n \\
                      & =               & [n^2pq + nqs + npr + rs]_n \\
                      & =               & [n^2pq]_n \oplus_n [nqs]_n \oplus_n [npr]_n \oplus_n [rs]_n \\
                      & =               & [rs]_n \\
                      & =               & r \odot_n s \\
                      & =               & [a]_n \odot_n [b]_n
    \end{IEEEeqnarray*}

\subsection{}

    \begin{IEEEeqnarray*}{rl}
          & \left(\left(3126 \oplus_{11} 21783\right) \otimes_{11} 213894\right) \oplus_{11} 31942 \\
        = & \left(\left[24909\right]_{11} \otimes_{11} 213894\right) \oplus_{11} 31942 \\
        = & \left(5 \otimes_{11} 213894\right) \oplus_{11} 31942 \\
        = & \left[1069470\right]_{11} \oplus_{11} 31942 \\
        = & 6 \oplus_{11} 31942 \\
        = & \left[31948\right]_{11} \\
        = & 4
    \end{IEEEeqnarray*}

\subsection{}

    \begin{IEEEeqnarray*}{rCltt}
      \left[2131897^8\right]_3   & = & \left[\prod_{i=1}^8 2131897\right]_3\\
                      & = & \prod_{i=1}^8 \left[2131897\right]_3   & \hspace{1cm} & Aufgabe 1.1 a) \\
                      & = & \prod_{i=1}^8 1             & \hspace{1cm} &
      \begin{IEEEeqnarraybox}[][c]{s}
         Es sind $21\cdot x$, $3\cdot x$, $18\cdot x$ \\
         und $9\cdot x$ durch $3$ teilbar, \\
         sodass nur $7$ für Rest sorgen kann\\
      \end{IEEEeqnarraybox} \\
                     & = & 1
    \end{IEEEeqnarray*}


\section{Die Skytale}

\subsection{}

\subsection{}

\subsection{}


\section{Cäsar\protect\raisebox{1ex}{\normalfont\tiny 2}}

\subsection{}

\subsection{}

\end{document}
